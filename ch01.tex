\documentclass{article}
\usepackage{graphicx} % Required for inserting images

\title{Proximity Service}
\author{xxx}
\date{March 2023}

\begin{document}

\maketitle

\section{Step1}
Understand Problem and Establish Design Scope\\
Functional Requirements:\\
\begin{itemize}
    \item return all businesses based on user lat log and radius
    \item update/add/delete business by business owner
    \item customer views detail about a business
\end{itemize}
Nonfunctional requirements:\\
\begin{itemize}
    \item low latency
    \item high availability and scalability
    \item data privacy
\end{itemize}
QPS estimation:\\
\begin{itemize}
    \item 100 million user per day
    \item 5 times per user
    \item business owner operation times negligible
    \item $QPS = \frac{100million * 5}{86400} = 5000$
\end{itemize}

\section{Step2}
\subsection{Propose High-level Design}
\begin{itemize}
    \item API Design\\
    RESTful API
    \item High-level Design\\
    users call location-based service\\
    business owners call business service\\
    \item algorithm for finding nearby services\\
    (expanded later)\\
    \item Data model\\
    it is read heavy model, so we can use relational db (mysql etc).\\
    using business\_id as key\\
\end{itemize}
\subsection{Expand on location-based service (LBS)}
stateless, easy to scale\\
\begin{itemize}
    \item two\_dimensional search\\
    use SQL WHERE to query for both lat and long\\
    WHERE Lat BETWEEN LatRange AND Long BETWEEN LongRange\\
    \begin{itemize}
        \item Good: Easy to implement and understand
        \item Bad: query involves too many data from db
    \end{itemize}
    \item evenly\_devided grid:\\
    devide all the world into fixed small grids\\
    \begin{itemize}
        \item Good: easy to understand and find the grid
        \item Bad: waste of memory, some grid in the sea, never used
    \end{itemize}
    \item GeoHash:\\
    divide using different precision\\
    biggest grid: 00, 01, 10, 11\\
    if a grid has too many business, divide it further:\\
    0101, 0100, 0110, 0111\\
    \begin{itemize}
        \item Good: easy to scale with precision (length of binary string)
        \item Bad: zoom level is fixed, has boundary issue:\\
        \begin{itemize}
            \item 0100 and 0001 are adjacent but do not share prefix
            \item if there is not enough business in the grid, we need to expand to the adjacent grid
            \item A solution: while in a grid, mathematically compute the adjacent 8 grids GeoHash value and add them all to the API, so won't miss anything
        \end{itemize}
    \end{itemize}
    \item QuadTree:\\
    in-memory structure, need to create a tree\\
    GeoHash is just a structless table\\
    Memory estimation:\\
    \begin{itemize}
        \item if 200 million business in our db\\
        \item each leaf node contains an area with 100 businesses\\
        \item each leaf node size ~800 bytes\\
        \item each internal node size ~100 bytes\\
        \item number of internal node is $\frac{1}{3}$ of leaf nodes\\
        \item in-memory storage is: $\frac{200 million}{100}* 800 + \frac{200 million}{100}*\frac{1}{3}*100 = 2GB$\\
        it can fit in memory well.\\
    \end{itemize}
    \begin{itemize}
        \item Good: fit it in the table and query
        \item Bad: Need to build at server start-up time, might take minutes\\
        if business add or delete, need to go down the tree and fix the node\\
    \end{itemize}
    \item Google S2\\
    Some mapping from map to 1D space\\
\end{itemize}
\section{Step3}
Design deep dive\\
\subsection{scale the database}
the db is small, do not need to shard\\
just use replica to replicate thru machines to ease read traffic\\
\subsection{use caching}
it can fit in memory\\
can use redis cache: ${geoHash, business_id}$ to get business detail for a specific user request
\end{document}
